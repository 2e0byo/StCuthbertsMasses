% This file defines the propert to be inserted into missalette.tex  In this way
% multiple feasts can be typeset very quickly.  missalette.tex should not normally
% need changing.  Note that this is not the most readable way to insert text
% into a LaTeX document, but it is the most powerful: the macros defined here
% are directly excecuted when building the document.

% For the title page
\newcommand{\feast}{ {{-cookiecutter.feast_name-}} }
% 
\newcommand{\masstype}{}
\newcommand{\introittranslation}{}
\newcommand{\collect}{%
  \l{}
  \e{}
}
\newcommand{\lessontranslation}{}
\newcommand{\responsorytranslation}{}
\newcommand{\gospel}{%
  \l{}
  \e{}
}
\newcommand{\offertorytranslation}{}
\newcommand{\communiontranslation}{}
\newcommand{\postcommunion}{%
  \l{}
  \e{}
}

\newcommand{\marian}{} % set to blank to suppress
% 


%%% Local Variables:
%%% mode: latex
%%% TeX-master: "missalette"
%%% End:
