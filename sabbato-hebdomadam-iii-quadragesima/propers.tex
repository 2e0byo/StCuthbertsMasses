% This file defines the propert to be inserted into missalette.tex  In this way
% multiple feasts can be typeset very quickly.  missalette.tex should not normally
% need changing.  Note that this is not the most readable way to insert text
% into a LaTeX document, but it is the most powerful: the macros defined here
% are directly excecuted when building the document.

% For the title page
\newcommand{\feast}{Sabbato post Dominicam III Quadragesimæ}
\newcommand{\masstype}{ Missa Cantata %
  % 
  Iuxta Ritum Dominicanum%
  % %
}
% 
\newcommand{\office}{%
  Verba mea auribus percipe Domine, intellige clamorem meum: intende voci orationis meae.  Rex meus, et Deus meus.
}
\newcommand{\officeTranslation}{%
  O Lord, hear my words, understand my cry; hearken to the voice of my prayer.  O my King and my God.
}
% 
\newcommand{\collect}{%
  \l{%
    Praesta, quaesumus, omnipotens Deus, ut qui se, affligendo carnem, ab alimentis abstinent, sectando justitiam, a culpa jejunent.
  }
  \e{%
    Grant, we beseech You, almighty God, that Your people who morify their flesh by abstinence from food may, by following after justice, fast from sin.
  }
  \per
}
\newcommand{\lesson}{%
  \l{%
 Et erat vir habitans in Babylone, et nomen ejus Joakim:
 et accepit uxorem nomine Susannam, filiam Helciæ, pulchram nimis, et timentem Deum:
 parentes enim illius, cum essent justi, erudierunt filiam suam secundum legem Moysi.
 Erat autem Joakim dives valde, et erat ei pomarium vicinum domui suæ: et ad ipsum confluebant Judæi, eo quod esset honorabilior omnium.
 Et constituti sunt de populo duo senes judices in illo anno, de quibus locutus est Dominus: Quia egressa est iniquitas de Babylone a senioribus judicibus, qui videbantur regere populum.
 Isti frequentabant domum Joakim, et veniebant ad eos omnes qui habebant judicia.
 Cum autem populus revertisset per meridiem, ingrediebatur Susanna, et deambulabat in pomario viri sui.
 Et videbant eam senes quotidie ingredientem et deambulantem, et exarserunt in concupiscentiam ejus:
 et everterunt sensum suum, et declinaverunt oculos suos ut non viderent cælum, neque recordarentur judiciorum justorum.
 Erant ergo ambo vulnerati amore ejus, nec indicaverunt sibi vicissim dolorem suum:
 erubescebant enim indicare sibi concupiscentiam suam, volentes concumbere cum ea.
 Et observabant quotidie sollicitius videre eam. Dixitque alter ad alterum:
 Eamus domum, quia hora prandii est. Et egressi, recesserunt a se.
 Cumque revertissent, venerunt in unum: et sciscitantes ab invicem causam, confessi sunt concupiscentiam suam: et tunc in communi statuerunt tempus quando eam possent invenire solam.
 Factum est autem, cum observarent diem aptum, ingressa est aliquando sicut heri et nudiustertius, cum duabus solis puellis, voluitque lavari in pomario: æstus quippe erat:
 et non erat ibi quisquam, præter duos senes absconditos, et contemplantes eam.
 Dixit ergo puellis: Afferte mihi oleum, et smigmata, et ostia pomarii claudite, ut laver.
 Et fecerunt sicut præceperat: clauseruntque ostia pomarii, et egressæ sunt per posticum ut afferrent quæ jusserat; nesciebantque senes intus esse absconditos.
 Cum autem egressæ essent puellæ, surrexerunt duo senes, et accurrerunt ad eam, et dixerunt:
 Ecce ostia pomarii clausa sunt, et nemo nos videt, et nos in concupiscentia tui sumus: quam ob rem assentire nobis, et commiscere nobiscum.
 Quod si nolueris, dicemus contra te testimonium, quod fuerit tecum juvenis, et ob hanc causam emiseris puellas a te.
 Ingemuit Susanna, et ait: Angustiæ sunt mihi undique: si enim hoc egero, mors mihi est: si autem non egero, non effugiam manus vestras.
 Sed melius est mihi absque opere incidere in manus vestras, quam peccare in conspectu Domini.
 Et exclamavit voce magna Susanna: exclamaverunt autem et senes adversus eam.
 Et cucurrit unus ad ostia pomarii, et aperuit.
 Cum ergo audissent clamorem famuli domus in pomario, irruerunt per posticum ut viderent quidnam esset.
 Postquam autem senes locuti sunt, erubuerunt servi vehementer, quia numquam dictus fuerat sermo hujuscemodi de Susanna. Et facta est dies crastina.
 Cumque venisset populus ad Joakim virum ejus, venerunt et duo presbyteri, pleni iniqua cogitatione adversus Susannam ut interficerent eam.
 Et dixerunt coram populo: Mittite ad Susannam filiam Helciæ uxorem Joakim. Et statim miserunt.
 Et venit cum parentibus, et filiis, et universis cognatis suis.
 Porro Susanna erat delicata nimis, et pulchra specie.
 At iniqui illi jusserunt ut discooperiretur (erat enim cooperta), ut vel sic satiarentur decore ejus.
 Flebant igitur sui, et omnes qui noverant eam.
 Consurgentes autem duo presbyteri in medio populi, posuerunt manus suas super caput ejus.
 Quæ flens suspexit ad cælum: erat enim cor ejus fiduciam habens in Domino.
 Et dixerunt presbyteri: Cum deambularemus in pomario soli, ingressa est hæc cum duabus puellis: et clausit ostia pomarii, et dimisit a se puellas.
 Venitque ad eam adolescens, qui erat absconditus, et concubuit cum ea.
 Porro nos cum essemus in angulo pomarii, videntes iniquitatem, cucurrimus ad eos, et vidimus eos pariter commisceri.
 Et illum quidem non quivimus comprehendere, quia fortior nobis erat, et apertis ostiis exilivit:
 hanc autem cum apprehendissemus, interrogavimus, quisnam esset adolescens, et noluit indicare nobis: hujus rei testes sumus.
 Credidit eis multitudo quasi senibus et judicibus populi, et condemnaverunt eam ad mortem.
 Exclamavit autem voce magna Susanna, et dixit: Deus æterne, qui absconditorum es cognitor, qui nosti omnia antequam fiant,
 tu scis quoniam falsum testimonium tulerunt contra me: et ecce morior, cum nihil horum fecerim, quæ isti malitiose composuerunt adversum me.
 Exaudivit autem Dominus vocem ejus.
 Cumque duceretur ad mortem, suscitavit Dominus spiritum sanctum pueri junioris, cujus nomen Daniel:
 et exclamavit voce magna: Mundus ego sum a sanguine hujus.
 Et conversus omnis populus ad eum, dixit: Quis est iste sermo, quem tu locutus es?
 Qui cum staret in medio eorum, ait: Sic fatui filii Isræl, non judicantes, neque quod verum est cognoscentes, condemnastis filiam Isræl?
 revertimini ad judicium, quia falsum testimonium locuti sunt adversus eam.
 Reversus est ergo populus cum festinatione, et dixerunt ei senes: Veni, et sede in medio nostrum, et indica nobis: quia tibi Deus dedit honorem senectutis.
 Et dixit ad eos Daniel: Separate illos ab invicem procul, et dijudicabo eos.
 Cum ergo divisi essent alter ab altero, vocavit unum de eis, et dixit ad eum: Inveterate dierum malorum, nunc venerunt peccata tua, quæ operabaris prius:
 judicans judicia injusta, innocentes opprimens, et dimittens noxios, dicente Domino: Innocentem et justum non interficies.
 Nunc ergo, si vidisti eam, dic sub qua arbore videris eos colloquentes sibi. Qui ait: Sub schino.
 Dixit autem Daniel: Recte mentitus es in caput tuum: ecce enim angelus Dei, accepta sententia ab eo, scindet te medium.
 Et, amoto eo, jussit venire alium, et dixit ei: Semen Chanaan, et non Juda, species decepit te, et concupiscentia subvertit cor tuum:
 sic faciebatis filiabus Isræl, et illæ timentes loquebantur vobis: sed filia Juda non sustinuit iniquitatem vestram.
 Nunc ergo, dic mihi sub qua arbore comprehenderis eos loquentes sibi. Qui ait: Sub prino.
 Dixit autem ei Daniel: Recte mentitus es et tu in caput tuum: manet enim angelus Domini, gladium habens, ut secet te medium, et interficiat vos.
 Exclamavit itaque omnis cœtus voce magna, et benedixerunt Deum, qui salvat sperantes in se.
 Et consurrexerunt adversus duos presbyteros (convicerat enim eos Daniel ex ore suo falsum dixisse testimonium), feceruntque eis sicut male egerant adversus proximum,
 ut facerent secundum legem Moysi. Et interfecerunt eos, et salvatus est sanguis innoxius in die illa.
  }
  \e{%
 There was a man called Joakim living in Babylon,
married to one Susanna, daughter of Helcias. This was a woman of great beauty, and one that feared God,
so well had her parents, religious folk, schooled their daughter in the law of Moses.
A rich man was Joakim, and had a fruit-garden close to his house; and he was much visited by the Jews, among whom there was none more honoured than he.
There came a year in which those two elders of the people were appointed judges, of whom the Lord said, Wickedness has sprung up in Babylon, and the roots of it are those elders and judges who claim to rule the people.
These two were often at Joakim’s house, and all those who had disputes to settle appeared before them there.
At noon, when the common folk had returned home, Susanna would walk about in her husband’s garden,
and these two elders, who saw her go in and walk there day after day, fell to lusting after her.
Reason they dethroned, and turned away their eyes from the sight of heaven; its just awards they would fain have forgotten.
The love that tortured both, neither to other would disclose;
confess it for very shame they might not, this hankering after a woman’s favours;
yet day after day they seized the opportunity to have sight of her. A day came at last when one said to the other,
Home go we, it is dinner-time; and go they did, taking their several ways;
yet both returned hot-foot to their watching-place, and there met one another. So there was questioning on both sides, and out came the story of their lust; and now they made common cause; at a suitable time they would waylay her together, when she was alone.
They watched, then, for their opportunity; and she, as her custom was, went out one day with two of her maids, and had a mind to bathe, there in the garden, for it was summer weather,
and none was by except the two elders; and they were in hiding, watching her.
So she bade her servants go and bring her oil and soap, and shut the garden door while she was a-bathing.
Her whim was obeyed; shut the door of the garden they did, and went out by a back entrance to bring her what she had asked for; they knew nothing of the elders that were hiding there within.
And these two, as soon as the servants were gone, rose from their hiding-place and ran to her side.
See, they told her, the garden door is shut, and there is no witness by. We are both smitten with a desire for thy favours; come, then, let us enjoy thee.
Refuse, and we will bear witness that thou hadst a gallant here, and this was the reason thou wouldst rid thyself of thy hand-maidens’ company.
Whereupon Susanna groaned deeply; There is no escape for me, she said, either way. It is death if I consent, and if I refuse, I shall be at your mercy.
Let me rather fall into your power through no act of mine, than commit sin in the Lord’s sight.
With that, Susanna cried aloud, and the elders, too, began crying shame on her;
meanwhile, one of them ran to the garden door and opened it.
And now the servants of the house, hearing such outcry in the garden, came running in through the back entrance to know what was afoot;
and they were greatly abashed when the elders told their story; never before had Susanna been defamed thus.When the morrow came,
there was a throng of people in Joakim’s house, and the two elders were there, intent upon their malicious design against Susanna’s life.
They asked publicly that Susanna, daughter of Helcias and wife to Joakim, should be sent for; sent for she was,
and came out with her parents and her children and all her kindred.
So dainty she was, and so fair,
these two knaves would have her let down her veil, the better to enjoy the sight of her charms.
All her friends, all her acquaintances, were in tears.
Then the two elders rose amidst the throng, and laid their hands upon Susanna’s head,
while she, weeping, looked up to heaven, in token that her heart had not lost confidence in the Lord.
We were walking in the garden apart, said the elders, when this woman came out with two hand-maidens. She had the garden door shut close, and sent the maidens away;
whereupon a young man, who had been in hiding till then, came out and had his will with her.
We, from a nook in the garden, saw what foul deed was being done, and ran up close, so that we had full view of their dalliance;
but lay hold of the man we could not; he was too strong for us, opening the garden door and springing out.
The woman we caught, and asked her who her gallant was, but she would not tell us. To all this, we bear witness.
They were elders, they were judges of the people, and they persuaded the assembly, without more ado, to pass the death sentence.
Whereupon Susanna cried aloud, Eternal God, no secret is hidden from thee, nothing comes to pass without thy foreknowledge.
Thou knowest that these men have borne false witness against me; wilt thou let me die, a woman innocent of all the charges their malice has invented?
And the Lord listened to her plea;
even as she was being led off to her death, all at once he roused to utterance the holy spirit that dwelt in a young boy there, called Daniel.
This Daniel raised his voice and cried out, I will be no party to the death of this woman;
and when all the people turned upon him, asking what he meant,
he stood there in their midst, and said, Are you such fools, men of Israel, as to condemn an Israelite woman without trial, without investigation of the truth?
Go back to the place of judgement; the witness they have borne against her is false witness.
Eagerly enough the people went back, and the elders would have Daniel sit with them, such credit God had given him beyond his years.
He bade them part the two men, at a distance from each other, while he questioned them.
So parted they were, and when the first was summoned, thus Daniel greeted him: Grown so old in years, and years ill spent! Now, that past sinning of thine has found thee out,
a man that perverts justice, persecutes innocence, and lets the guilty go free. Has not the Lord said, Never shalt thou put the innocent man, the upright man, to death?
Thou foundest her; good; they met under a tree; tell us what kind of tree. And he answered, Under a mastic-tree I surprised them.
The right word! cried Daniel; prized asunder thyself shall be, when God bids his angel requite thee for this calumny.
Then he had this one removed, and bade the other come near. Brood of Chanaan, said he, and no true son of Juda, so beauty ensnared thee? So lust drove thy heart astray?
Such approaches you have made, long since, to women of the other tribes, and they, from very fear, admitted your suit; but you could not bring a woman of Juda to fall in with your wicked design.
And now tell me, under what tree it was thou didst find them talking together? Under a holm-oak, said he, I saw them.
The right word again! cried Daniel. Saw thee asunder the angel of the Lord will, with the sharp blade he carries yonder; you are both dead men.
And with that, the whole multitude cried aloud, blessing God that is the deliverer of those who trust in him.
And they turned on the two elders, by Daniel’s questioning self-convicted of false witness; served they must be as they would have served others,
and the law of Moses obeyed; so they put them to death. That day, an innocent life was saved.
  }
}

% 
\newcommand{\responsory}{%
  Si ambulem in medio umbrae mortis, non timebo mala: quoniam tu mecum es, Domine.  Virga tua et baculus tuus, ipsa me consolata sunt.
}
\newcommand{\responsoryTranslation}{%
  If I should walk in the midst of the shadow of death I will fear no evils, for You are with me, O lord.  Your rod and Your staff have comforted me.
}
% 
\newcommand{\gospel}{%
  \l{%
 Jesus autem perrexit in montem Oliveti:
 et diluculo iterum venit in templum, et omnis populus venit ad eum, et sedens docebat eos.
 Adducunt autem scribæ et pharisæi mulierem in adulterio deprehensam: et statuerunt eam in medio,
 et dixerunt ei: Magister, hæc mulier modo deprehensa est in adulterio.
 In lege autem Moyses mandavit nobis hujusmodi lapidare. Tu ergo quid dicis?
 Hoc autem dicebant tentantes eum, ut possent accusare eum. Jesus autem inclinans se deorsum, digito scribebat in terra.
 Cum ergo perseverarent interrogantes eum, erexit se, et dixit eis: Qui sine peccato est vestrum, primus in illam lapidem mittat.
 Et iterum se inclinans, scribebat in terra.
 Audientes autem unus post unum exibant, incipientes a senioribus: et remansit solus Jesus, et mulier in medio stans.
 Erigens autem se Jesus, dixit ei: Mulier, ubi sunt qui te accusabant? nemo te condemnavit?
 Quæ dixit: Nemo, Domine. Dixit autem Jesus: Nec ego te condemnabo: vade, et jam amplius noli peccare.
  }
  \e{%
	Jesus meanwhile went to the mount of Olives.
And at early morning he appeared again in the temple; all the common folk came to him, and he sat down there and began to teach them.
And now the scribes and Pharisees brought to him a woman who had been found committing adultery, and made her stand there in full view;
Master, they said, this woman has been caught in the act of adultery.
Moses, in his law, prescribed that such persons should be stoned to death; what of thee? What is thy sentence?
They said this to put him to the test, hoping to find a charge to bring against him. But Jesus bent down, and began writing on the ground with his finger.
When he found that they continued to question him, he looked up and said to them, Whichever of you is free from sin shall cast the first stone at her.
Then he bent down again, and went on writing on the ground.
And they began to go out one by one, beginning with the eldest, till Jesus was left alone with the woman, still standing in full view.
Then Jesus looked up, and asked her, Woman, where are thy accusers? Has no one condemned thee?
No one, Lord, she said. And Jesus said to her, I will not condemn thee either. Go, and do not sin again henceforward.
  }
}
\newcommand{\offertory}{%
  Gressus meos dirige secundum eloquium tuum: ut non dominetur mei omnis injustitia, Domine.
}
\newcommand{\offertoryTranslation}{%
  Direct my steps according to Your word, that no iniquity may have dominion over me, O Lord.
}
\newcommand{\secret}{%
  \l{%
    Concede quaesumus omnipotens Deus, ut hujus sacrificii munus oblatum, fragilitatem nostram ab omni malo purget semper et muniat.
  }
  \e{%
    Grant, we pray, almighty God, that the gift of this sacrifice now offered to You, may ever defend us in our frailty and cleanse us from every ill: through our Lord.
  }
}
\newcommand{\communion}{%
  Nemo te condemnavit, mulier?  Nemo, Domine.  Nec ego te condemnabo: jam amplius noli peccare.
}
\newcommand{\communionTranslation}{%
  Has no one condemned you, woman?  No one, Lord.  Neither will I condemn you; now sin no more.
}
\newcommand{\postcommunion}{%
  \l{%
    Quaesumus, omnipotens Deus, ut inter ejus membra numeremur, cujus corpori communicamus et sanguini.  Qui tecum vivt et regnat.
  }
  \e{%
    We beseech You, almighty God, that we may be counted among the members of Him of whose body and blood we have partaken: who lives and reigns with You.
  }
}

% File paths: we don't use symlinks as (a) not all platforms support them, and
% (b) they don't fit nicely with the flow we're using.

\newcommand{\kyriePath}{../DominicanOrdinaries/masses/6/kyrie}
\newcommand{\gloriaPath}{../DominicanOrdinaries/masses/6/gloria}
\newcommand{\sanctusPath}{../DominicanOrdinaries/masses/6/sanctus}
\newcommand{\agnusPath}{../DominicanOrdinaries/masses/6/agnus}
\newcommand{\itePath}{../DominicanOrdinaries/masses/6/ite}
% 
%%% Local Variables:
%%% mode: latex
%%% TeX-master: "missalette"
%%% End:
